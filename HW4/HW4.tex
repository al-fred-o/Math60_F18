\documentclass[12pt,letterpaper]{hmcpset}
\usepackage[margin=1in]{geometry}
\usepackage{graphicx}

% info for header block in upper right hand corner
\name{Box number: \underline{\hspace{1cm}} (name on back) }
\class{Math 60: Section \underline{\hspace{1cm}}}
\assignment{Assignment 4}
\duedate{Due Date: 10/5/18}

\begin{document}

\problemlist{
Section 4.1: 9, 10, 20, 28, 34\\
Section 4.2: 6, 12, 22a, 28, 30, 34, 46\\
Section 4.3: 6, 24, 26
}

\begin{problem}
Section 4.1 \#9. In Exercises 8-15, find the first- and second-order Taylor polynomials for the given function $f$ at the given point $\textbf{a}$.\\
$$ f(x, y) = 1/(x^2 + y^2 + 1), \textbf{a} = (1, -1) $$
\end{problem}

\newpage

\begin{problem}
Section 4.1 \#10. In Exercises 8-15, find the first- and second-order Taylor polynomials for the given function $f$ at the given point $\textbf{a}$.\\
$$ f(x, y) = e^{2x+y}, \textbf{a} = (0, 0) $$
\end{problem}

\newpage

\begin{problem}
Section 4.1 \#20. In Exercises 16-20, calculate the Hessian matrix $Hf( \textbf{a} )$ for the indicated function $f$ at the indicated point \textbf{a}.\\
$$ f(x, y, z) = e^{2x-3y} \sin{5z}, \textbf{a} = (0, 0, 0) $$
\end{problem}

\newpage

\begin{problem}
Section 4.1 \#28. Determine the total differential of the functions given in Exercises 28-32.\\
$$ f(x, y) = x^2 y^3$$
\end{problem}

\newpage

\begin{problem}
Section 4.1 \#34. Near the point $(1, -2, 1)$, is the function $g(x, y, z) = x^3 - 2xy + x^2 z + 7z$ most sensitive to changes in $x$, $y$, or $z$?
\end{problem}

\newpage

\begin{problem}
Section 4.2 \#6. In Exercises 3-20, identify and determine the nature of the critical points of the given functions.\\
$$ f(x, y) = y^4 - 2xy^2 + x^3 - x $$
\end{problem}

\newpage

\begin{problem}
Section 4.2 \#12. In Exercises 3-20, identify and determine the nature of the critical points of the given functions.\\
$$ f(x, y) = e^{-x} (x^2 + 3y^2) $$
\end{problem}

\newpage

\begin{problem}
Section 4.2 \#22a. Under what conditions on the constant $k$ will the function
$$ f(x, y) = kx^2 - 2xy + ky^2 $$
have a nondegenerate local minimum at $(0, 0)$? What about a local maximum?
\end{problem}

\newpage

\begin{problem}
Section 4.2 \#28. Show that the largest rectangular box having a fixed surface area must be a cube.
\end{problem}

\newpage

\begin{problem}
Section 4.2 \#30. Find the points on the surface $xy + z^2 = 4$ that are closest to the origin. Be sure to give a convincing argument that your answer is correct.
\end{problem}

\newpage

\begin{problem}
Section 4.2 \#34. A metal plate has the shape of the region $x^2 + y^2 \leq 1$. The plate is heated so that the temperature at any point $(x, y)$ on it is indicated by
$$ T(x, y) = 2x^2 + y^2 - y + 3. $$
Find the hottest and coldest points on the plate and the temperature at each of these points. (Hint: Parametrize the boundary of the plate in order to find any critical points there.)
\end{problem}

\newpage

\begin{problem}
Section 4.2 \#46. In Exercises 46-48, (a) find all critical points of the given function $f$ and identify their nature as local extrema and (b) determine, with explanation, any global extrema of $f$.\\
$$ f(x, y) = e^{x^2 + 5y^2} $$
\end{problem}

\newpage

\begin{problem}
Section 4.3 \#6. In Exercises 2-12, use Lagrange multipliers to identify the critical points of $f$ subject to the given constraints.\\
$$ f(x, y, z) = x^2 + y^2 + z^2, x + y - z = 1 $$
\end{problem}

\newpage

\begin{problem}
Section 4.3 \#24. You are sending a birthday present to your calculus instructor. Fly-By-Night Delivery Service insists that any package it ships be such that the sum of the length plus the girth be at most $108$ in. (The girth is the perimeter of the cross section perpendicular to the length axis--see Figure 4.31.) What are the dimensions of the largest present you can send?
\end{problem}

\newpage

\begin{problem}
Section 4.3 \#26. An industrious farmer is designing a silo to hold her $900 \pi$ $\text{ft}^3$ supply of grain. The silo is to be cylindrical in shape with a hemispherical roof. (See Figure 4.32.) Suppose that it costs five times as much (per square foot of sheet metal used) to fashion the roof of the silo as it does to make the cylindrical floor and twice as much to make the cylindrical walls as the floor. If you were to act as consultant for this project, what dimensions would your recommend so that the \textit{total} cost would be a minimum? On what do you base your recommendation? (Assume that the entire silo can be filled with grain.)
\end{problem}
\end{document}
